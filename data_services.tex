% MMW: I'd order these subsections roughly in order of the software stack... most fundamental to most complex

\todo{Describe performance portability... I think this should go into the intro section.}

\todo{@Siva: do this!}

\subsection{Kokkos Kernels}\label{subsec:kk}
Kokkos Kernels~\cite{rajamanickam2021kokkoskernels} is part of the Kokkos ecosystem
~\cite{trott2021kokkos} and provides node local implementations of mathematical kernels
widely used across packages in Trilinos. As a member of the Kokkos
ecosystem, Kokkos Kernels is tightly integrated on Kokkos features and aims at
delivering performance portable algorithms across major CPU and GPU based HPC systems.
Due to its node local nature, Kokkos Kernels does not rely on MPI or other communication
libraries unlike numerous other packages in Trilinos.

The implementation of Kokkos Kernels algorithms leverage the hierarchical parallelism
exposed by the Kokkos library~\cite{kim2017designing} and increasingly provides coverage
for stream callable kernels. To ensure flexibility for the distributed libraries that
might call its algorithms, Kokkos Kernels provides thread safe and asynchronous
implementations for most of its kernels. Kokkos Kernels also serves as a major point of
integration for vendor optimized libraries such as cuBLAS, cuSPARSE, rocBLAS, rocSPARSE,
MKL, ARMpl and others.

The capabilities that Kokkos Kernels provides can be divided in four major categories:
1. BLAS algorithms, 2. sparse linear algebra and preconditioners, 3. graph algorithms, and
4. batched dense and sparse linear algebra~\cite{liegeois2023performance}. The main
points of integration of Kokkos Kernels in Trilinos are Tpetra for the dense and sparse
linear algebra capabilities, Ifpack2 for the preconditioners and batched algorithms,
the multigrid package MueLu that relies on these features both directly and indirectly
as well as on some specialized algorithms such as graph
coloring/coarsening~\cite{kelley2022parallel} and fused Jacobi-SpGEMM (generalized sparse matrix matrix multiply) kernels.

Similarly to the Kokkos library, Kokkos Kernels is developed in its own GitHub
repository\footnote{https://github.com/kokkos/kokkos-kernels} outside of the Trilinos
GitHub repository. Every version of the library is integrated and tested in Trilinos
as part of the Kokkos ecosystem release process. Additional information on Kokkos
Kernels capabilities can be found here~\cite{deveci2018multithreaded,wolf2017fast}.

\todo{I'm not sure if we want to include Teuchos, but it might be necessary since it is omnipresent in the use of Trilinos.  We probably should expand this a bit more.}

% MMW I don't think Teuchos should be under framework.  I think we want to stick to logical groupings versus load-balanced ones.
\subsection{Teuchos}

Teuchos provides a suite of common tools for many Trilinos packages. These tools include memory management classes~\cite{bartlett2010} such as ``smart'' pointers and arrays, ``parameter lists'' for communicating hierarchical lists of parameters between library or application layers, templated wrappers for the BLAS and LAPACK, XML parsers, and other utilities. They provide a unified ``look and feel'' across Trilinos packages, and help avoid common programming mistakes.



\subsection{Tpetra}\label{subsec:tpetra}
Tpetra \cite{hoemmen2015tpetra} provides the distributed-memory
infrastructure for sparse linear algebra computations.  It implements
distributed-memory linear algebra objects, such as sparse graphs,
sparse matrices and dense vectors, using Kokkos for local data
storage.  Distributed-memory sparse linear algebra operations, such as
a sparse matrix-vector product, are implemented through on-node calls
to Kokkos Kernels and inter-node MPI communication.   Tpetra features
include:
\begin{itemize}
\item \textit{Maps} --- distributions of objects over MPI ranks.
\item \textit{(Multi)Vectors} --- storage of dense vectors or collections of
vectors (multivectors) and associated BLAS-1 like kernels (e.g., dot
products, norms, scaling, vector addition, pointwise vector
multiplication) as well as tall skinny QR (TSQR)  factorization for multivectors.
\item \textit{Import/export} --- moving vector, graph and matrix data
between different distributions (maps).  This is key for performing
halo/boundary exchanges, as well as other kernels such as
sparse matrix-matrix multiplication.
\item \textit{Sparse graphs} --- in compressed sparse row (CSR)
format.  Graphs also include import/export objects for use in
halo/boundary exchanges associate with the graph.
\item \textit{Sparse matrices} --- in compressed sparse row (CSR) and
block compressed sparse row (BSR) formats.  Associated kernels include
sparse matrix vector product (SPMV), sparse matrix-matrix
multiplication (SPGEMM) and triple-product, sparse matrix-matrix addition, sparse matrix transpose, diagonal extraction,
Frobenius norm calculation, and row/column scaling.
\end{itemize}

\todo{@Brian/Chris: do you want to expand this a bit?}

\subsection{Thyra}
The Thyra library provides abstractions for use in writing high level abstract numerical algorithms (ANAs). Thyra is used by other Trilinos packages for implementing ANAs such as solvers for linear equations, nonlinear equations, bifurcation and stability analysis, optimization and uncertainty quantification. The abstractions hide concrete linear algebra and solver implementations so that the algorithms can be written once and reused across many application codes. For example, one could write a Krylov solver and it could be used with multiple concrete linear algebra implementations such as Tpetra in Trilinos, Epetra in Trilinos or the vector/matrix objects in PETSc.

Thyra is organized into three sets of abstractions depending on the type of operation required. All are templated on the scalar type. The sets are:
\begin{itemize}
\item \emph{Vector/Operator Interfaces:} The first level is an abstraction for vectors and operators \cite{Bartlett2007}. This set includes abstractions for vector spaces, vectors, multivectors and linear operators. Thyra provides a default set of vector/vector and vector-operator operations for standard linear algebra tasks such ad AXPY from BLAS. Defining an extensible interface such that users can add arbitrary vector/matrix operations and maintain performance can be difficult. A general tool to achieve this is the Trilinos utility package called RTOp \cite{rtop}. RTOp provides an interface for users to add new Thyra operations. RTOp accepts a vector of Thyra input objects and a vector of Thyra output objects along with a functor. The RTOp will apply the functor to each element of the data.

\item \emph{Operator Solve Interfaces:} The second set of abstractions supply abstract linear solvers that can be attached to operators. The interfaces include base classes for preconditioners, preconditioner factories, linear operators with a solver, and factories for linear operators with a solver. The factories build and update the preconditioners and linear solvers as needed. The solver interfaces allow one to wrap any concrete linear solver or preconditioner for use with the ANAs. Trilinos provides a separate library named Stratimikos that wraps all of the Trilinos preconditioners and linear solvers in Thyra abstractions. This library is driven by the Teuchos::ParameterList options.

\item \emph{Nonlinear Interfaces:} The final set of abstractions support nonlinear operations. Thyra provides an interface that wraps nonlinear solvers. Implicit time integration and optimization solvers typically require nonlinear solves and can abstract away the concrete implementation via Thyra. The NOX package in Trilinos provides a concrete implementation of this interface. The final interface in Thyra is for querying a physics application for common objects required by high level Newton-based solvers. The Thyra::ModelEvaluator interface allows codes to ask the application to evaluate residuals, Jacobians, parametric sensitivities, Hessians and post processed quantities/derivatives of interest. The interface is designed to be stateless (i.e. consecutive calls with the same inputs should produce the same outputs). This design was chosen to avoid pitfalls in complex application codes interacting with general nonlinear algorithms. The design does not preclude its use in stateful applications that have a ``memory'' between time steps, but rather forces the explicit use of the stateful information in the interface.
\end{itemize}



