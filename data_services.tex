% !TEX root = main.tex

% MMW: I'd order these subsections roughly in order of the software stack... most fundamental to most complex
The Core product area provides essential tools for managing and distributing data across processing elements in parallel computing environments, addressing challenges such as load balancing and data redistribution. Key components discussed here include the Kokkos performance portability framework, Teuchos tools, Tpetra distributed-memory linear algebra, Thyra abstractions for high-level formulations of numerical algorithms, and Zoltan2 for combinatorial algorithms in parallel computing.

\subsection{Performance Portability: Kokkos Core}\label{subsec:kokkos}
Kokkos Core\footnote{For historical reasons the package name is ``Kokkos'' since it precedes the creation of other Kokkos subprojects such as Kokkos Kernels.} is a programming model designed for performance portability across hardware architectures such as CPUs and GPUs. Two primary insights into achieving performance portability shaped Kokkos' development:
i) abstractions for both parallel execution and data structures are required, and
ii) the fundamental programming model must be descriptive rather than prescriptive.

The Kokkos programming model includes \texttt{execution spaces} to specify where an algorithm should run, \texttt{execution policies} to indicate how the algorithm should be run in terms of hardware resources, and \texttt{execution patterns} to designate whether the \texttt{for}, \texttt{reduce}, or \texttt{scan} pattern will be used by the algorithm. The data structure abstractions are built around the concepts of \texttt{memory spaces} to indicate where memory is allocated, \texttt{memory layouts} to specify how data is laid out in memory, and \texttt{memory traits} that enable customizations of how data is accessed. The primary data structure incorporating
these concepts is \texttt{Kokkos::View}, a multidimensional array abstraction that inspired
the ISO C++23 \texttt{mdspan} class\footnote{\url{https://www.open-std.org/jtc1/sc22/wg21/docs/papers/2022/p0009r18.html}}. Detailed descriptions of Kokkos Core can be found
in~\cite{edwards2014} and \cite{trott2022kokkos}.

\subsection{Performance Portable Kernels: Kokkos Kernels}\label{subsec:kk}
Kokkos Kernels~\cite{rajamanickam2021kokkoskernels} is part of the Kokkos ecosystem~\cite{trott2021kokkos} and provides node-local implementations of mathematical kernels widely used across packages in Trilinos. As a member of the Kokkos ecosystem, Kokkos Kernels tightly integrates Kokkos features and aims at delivering performance\hyp{}portable algorithms across major CPU- and GPU-based HPC systems. Due to its node-local nature, Kokkos Kernels does not rely on MPI or other communication libraries, unlike numerous other packages in Trilinos.

The implementations of Kokkos Kernels algorithms leverage the hierarchical parallelism exposed by the Kokkos library~\cite{kim2017designing}. To ensure flexibility for the distributed libraries that might call its algorithms, Kokkos Kernels provides thread-safe implementations for most of its kernels, with increasing coverage for asynchronous implementations that allow an execution space instance to be passed when calling a kernel. Kokkos Kernels also serves as a major point of integration for vendor optimized libraries such as cuBLAS, cuSPARSE, rocBLAS, rocSPARSE, MKL, ARMpl and others.

The capabilities provided by Kokkos Kernels can be divided into four major categories:
1. BLAS algorithms, 2. sparse linear algebra and preconditioners, 3. graph algorithms, and
4. batched dense and sparse linear algebra~\cite{liegeois2023performance}. The main
points of integration of Kokkos Kernels in Trilinos are Tpetra for the dense and sparse
linear algebra capabilities, Ifpack2 for the preconditioners and batched algorithms,
and the multigrid package MueLu which relies directly and indirectly on these general capabilities as well as more specialized algorithms such as graph coloring/coarsening~\cite{kelley2022parallel} and fused Jacobi-SpGEMM (generalized sparse matrix-matrix multiply) kernels.

Similar to the Kokkos library, Kokkos Kernels is developed in its own GitHub
repository\footnote{https://github.com/kokkos/kokkos-kernels} outside of the Trilinos
GitHub repository. Every version of the library is integrated and tested in Trilinos
as part of the Kokkos ecosystem release process. Additional information on Kokkos
Kernels' capabilities can be found here~\cite{deveci2018multithreaded,wolf2017fast}.

%\todo{I'm not sure if we want to include Teuchos, but it might be necessary since it is omnipresent in the use of Trilinos.  We probably should expand this a bit more.}

% MMW I don't think Teuchos should be under framework.  I think we want to stick to logical groupings versus load-balanced ones.

\subsection{Tools: Teuchos}

Teuchos provides a suite of common tools for many Trilinos packages. These tools include memory management classes~\cite{bartlett2010} such as smart pointers and arrays, parameter lists for communicating hierarchical lists of parameters between library or application layers, templated wrappers for BLAS and LAPACK, XML parsers, MPI, and other utilities. They provide a unified ``look and feel'' across Trilinos packages and help avoid common programming mistakes.



\subsection{Distributed-Memory Linear Algebra: Tpetra}\label{subsec:tpetra}
Tpetra~\cite{Baker2012,hoemmen2015tpetra} provides the distributed-memory
infrastructure for linear algebra objects such as sparse graphs,
sparse matrices, and dense vectors. These linear algebra objects are
implemented using MPI-based abstractions for the inter-process communication,
and Kokkos-based data structures for the portions of the object
owned by each process (MPI rank). Kokkos Kernels is used for
performing local, parallel computations within each MPI rank.

The provided objects are templated on scalar type (e.g., \code{double}, \code{float}, or a Sacado or Stokhos type), local index type, global index type and Kokkos device (pair of execution and memory spaces).
The distribution of the linear algebra objects is described by a \code{Map} data structure, which defines which elements are present on which ranks.
Here ``element'' could be a row or column of a sparse matrix, or an entry of a vector.
Global indices number elements across an entire distributed object, while local indices number elements only within a rank.
Since memory capacity limits the sizes of local objects, local indices can be safely represented with a more compact type than global indices (e.g., 32-bit \code{int} instead of 64-bit \code{long}).

\code{DistObject} is Tpetra's abstract base class for general distributed objects.
Subclasses define how to pack elements into a buffer for sending with MPI, and how to unpack received elements.
The \code{Import} and \code{Export} classes describe how one \code{DistObject} communicates data to another.
They are both created from a pair of \code{Map}s, a source and a target. For example, halo exchange on a vector is an \code{Import} where
the source is the vector's original distribution, and the target is the same except that each rank
also ``owns'' the degrees-of-freedom in the halo around its subdomain.

The following are the primary linear algebra objects in Tpetra (all are subclasses of \code{DistObject}):
\begin{itemize}
\item \emph{(Multi)Vectors:} storage of dense vectors or collections of
vectors (multivectors) and associated BLAS-1 like kernels (e.g., dot
products, norms, scaling, vector addition, pointwise vector
multiplication) as well as tall skinny QR (TSQR) factorization for multivectors.
\item \emph{CrsGraph} a sparse graph in compressed row storage (CRS)
format. Graphs also include import/export objects for use in
halo/boundary exchanges associated with the graph.
\item \emph{CrsMatrix} and \emph{BlockCrsMatrix} are sparse matrices in CRS and
block compressed sparse row (BSR) formats. Associated kernels include
sparse matrix vector product (SPMV), sparse matrix-matrix
multiplication (SPGEMM) and triple-product, sparse matrix-matrix addition, sparse matrix transpose, diagonal extraction,
Frobenius norm calculation, and row/column scaling.
\end{itemize}

\subsection{Abstract Numerical Algorithms: Thyra}
The Thyra package provides abstractions for implementing high-level abstract numerical algorithms (ANAs). Thyra is used by other Trilinos packages for implementing ANAs such as solvers for linear and nonlinear equations, bifurcation and stability analysis, optimization, and uncertainty quantification. The abstractions hide concrete linear algebra and solver implementations so that the algorithms can be written once and reused across many application codes. For example, one could write a Krylov solver and use it with multiple concrete linear algebra implementations such as Tpetra in Trilinos %, Epetra in Trilinos
or the vector/matrix objects in PETSc~\cite{petsc-web-page}.

Thyra is organized into three sets of abstractions depending on the type of operation required. All are templated on the scalar type. These sets are:
\begin{itemize}
\item \emph{Vector/Operator Interfaces:} The first level is an abstraction for vectors and operators \cite{Bartlett2007}. This set includes abstractions for vector spaces, vectors, multivectors, and linear operators. Thyra provides a default set of vector/vector and vector-operator operations for standard linear algebra tasks such as AXPY from BLAS. Defining an extensible interface such that users can add arbitrary vector/matrix operations and maintain performance can be difficult. A general tool to achieve this is the Trilinos utility package RTOp \cite{rtop}, which provides an interface for users to add new Thyra operations. RTOp accepts a vector of Thyra input objects and a vector of Thyra output objects along with a functor. RTOp will then apply the functor to each element of the data.

\item \emph{Operator Solve Interfaces:} The second set of abstractions supplies abstract linear solvers that can be attached to operators. The interfaces include base classes for preconditioners, preconditioner factories, linear operators with a solver, and factories for linear operators with a solver. The factories build and update the preconditioners and linear solvers as needed. The solver interfaces allow one to wrap any concrete linear solver or preconditioner for use with the ANAs. Trilinos provides a separate library named Stratimikos that wraps all of the Trilinos preconditioners and linear solvers in Thyra abstractions. Run-time configuration of Stratimikos is driven by \code{Teuchos::ParameterList} options.
Details on Stratimikos are given in Section~\ref{sec:Stratimikos}.

\item \emph{Nonlinear Interfaces:} The final set of abstractions supports nonlinear operations. Thyra provides an interface that wraps nonlinear solvers. Implicit time integration and optimization solvers typically require nonlinear solves and can abstract away the concrete implementation via Thyra. The NOX package in Trilinos (described in Section~\ref{sec:nox}) provides a concrete implementation of this interface. The final interface in Thyra is for querying a physics application for common objects required by high-level Newton-based solvers. The \code{Thyra::ModelEvaluator} interface allows codes to ask the application to evaluate residuals, Jacobians, parametric sensitivities, Hessians, and post-processed quantities/derivatives of interest. The interface is designed to be stateless (i.e., consecutive calls with the same inputs should produce the same outputs). This design was chosen to avoid pitfalls in complex application codes interacting with general nonlinear algorithms. The design does not preclude its use in stateful applications that have a ``memory'' between timesteps, but rather forces the explicit use of the stateful information in the interface.
\end{itemize}


\subsection{Combinatorial algorithms with hybrid parallelism: Zoltan2}
Zoltan2 is a package for partitioning, load balancing, and other combinatorial scientific computing. It can be viewed as a successor to Zoltan~\cite{Boman2012} that provides both tighter integration with other Trilinos packages and support for hybrid parallelism. It is used primarily to partition and distribute data objects (such as sparse matrices) across processors.

Zoltan2 has a highly modular design. The user provides input via
input adapters. Adapters for sparse graphs and matrices are provided for Xpetra, which can also be used to interface Tpetra data structures.
The models create a graph, hypergraph, or coordinate model from the user input, which is typically a sparse matrix, mesh, or vector.
The algorithms then use the model to solve the particular problem (e.g., partitioning or coloring) requested by the user.

Zoltan2 supports both distributed-memory parallelism and on-node parallelism for both CPU and GPU (via Kokkos).
Algorithms that support such hybrid parallelism include Multi-jagged~\cite{Z1} (geometric partitioning), Sphynx~\cite{Z2} (spectral graph partitioning), graph coloring~\cite{Z5}, and MPI task placement~\cite{Z3}.
